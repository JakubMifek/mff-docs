
% Default to the notebook output style

    


% Inherit from the specified cell style.




    
\documentclass[11pt]{article}

    
    
    \usepackage[T1]{fontenc}
    % Nicer default font (+ math font) than Computer Modern for most use cases
    \usepackage{mathpazo}

    % Basic figure setup, for now with no caption control since it's done
    % automatically by Pandoc (which extracts ![](path) syntax from Markdown).
    \usepackage{graphicx}
    % We will generate all images so they have a width \maxwidth. This means
    % that they will get their normal width if they fit onto the page, but
    % are scaled down if they would overflow the margins.
    \makeatletter
    \def\maxwidth{\ifdim\Gin@nat@width>\linewidth\linewidth
    \else\Gin@nat@width\fi}
    \makeatother
    \let\Oldincludegraphics\includegraphics
    % Set max figure width to be 80% of text width, for now hardcoded.
    \renewcommand{\includegraphics}[1]{\Oldincludegraphics[width=.8\maxwidth]{#1}}
    % Ensure that by default, figures have no caption (until we provide a
    % proper Figure object with a Caption API and a way to capture that
    % in the conversion process - todo).
    \usepackage{caption}
    \DeclareCaptionLabelFormat{nolabel}{}
    \captionsetup{labelformat=nolabel}

    \usepackage{adjustbox} % Used to constrain images to a maximum size 
    \usepackage{xcolor} % Allow colors to be defined
    \usepackage{enumerate} % Needed for markdown enumerations to work
    \usepackage{geometry} % Used to adjust the document margins
    \usepackage{amsmath} % Equations
    \usepackage{amssymb} % Equations
    \usepackage{textcomp} % defines textquotesingle
    % Hack from http://tex.stackexchange.com/a/47451/13684:
    \AtBeginDocument{%
        \def\PYZsq{\textquotesingle}% Upright quotes in Pygmentized code
    }
    \usepackage{upquote} % Upright quotes for verbatim code
    \usepackage{eurosym} % defines \euro
    \usepackage[mathletters]{ucs} % Extended unicode (utf-8) support
    \usepackage[utf8x]{inputenc} % Allow utf-8 characters in the tex document
    \usepackage{fancyvrb} % verbatim replacement that allows latex
    \usepackage{grffile} % extends the file name processing of package graphics 
                         % to support a larger range 
    % The hyperref package gives us a pdf with properly built
    % internal navigation ('pdf bookmarks' for the table of contents,
    % internal cross-reference links, web links for URLs, etc.)
    \usepackage{hyperref}
    \usepackage{longtable} % longtable support required by pandoc >1.10
    \usepackage{booktabs}  % table support for pandoc > 1.12.2
    \usepackage[inline]{enumitem} % IRkernel/repr support (it uses the enumerate* environment)
    \usepackage[normalem]{ulem} % ulem is needed to support strikethroughs (\sout)
                                % normalem makes italics be italics, not underlines
    

    
    
    % Colors for the hyperref package
    \definecolor{urlcolor}{rgb}{0,.145,.698}
    \definecolor{linkcolor}{rgb}{.71,0.21,0.01}
    \definecolor{citecolor}{rgb}{.12,.54,.11}

    % ANSI colors
    \definecolor{ansi-black}{HTML}{3E424D}
    \definecolor{ansi-black-intense}{HTML}{282C36}
    \definecolor{ansi-red}{HTML}{E75C58}
    \definecolor{ansi-red-intense}{HTML}{B22B31}
    \definecolor{ansi-green}{HTML}{00A250}
    \definecolor{ansi-green-intense}{HTML}{007427}
    \definecolor{ansi-yellow}{HTML}{DDB62B}
    \definecolor{ansi-yellow-intense}{HTML}{B27D12}
    \definecolor{ansi-blue}{HTML}{208FFB}
    \definecolor{ansi-blue-intense}{HTML}{0065CA}
    \definecolor{ansi-magenta}{HTML}{D160C4}
    \definecolor{ansi-magenta-intense}{HTML}{A03196}
    \definecolor{ansi-cyan}{HTML}{60C6C8}
    \definecolor{ansi-cyan-intense}{HTML}{258F8F}
    \definecolor{ansi-white}{HTML}{C5C1B4}
    \definecolor{ansi-white-intense}{HTML}{A1A6B2}

    % commands and environments needed by pandoc snippets
    % extracted from the output of `pandoc -s`
    \providecommand{\tightlist}{%
      \setlength{\itemsep}{0pt}\setlength{\parskip}{0pt}}
    \DefineVerbatimEnvironment{Highlighting}{Verbatim}{commandchars=\\\{\}}
    % Add ',fontsize=\small' for more characters per line
    \newenvironment{Shaded}{}{}
    \newcommand{\KeywordTok}[1]{\textcolor[rgb]{0.00,0.44,0.13}{\textbf{{#1}}}}
    \newcommand{\DataTypeTok}[1]{\textcolor[rgb]{0.56,0.13,0.00}{{#1}}}
    \newcommand{\DecValTok}[1]{\textcolor[rgb]{0.25,0.63,0.44}{{#1}}}
    \newcommand{\BaseNTok}[1]{\textcolor[rgb]{0.25,0.63,0.44}{{#1}}}
    \newcommand{\FloatTok}[1]{\textcolor[rgb]{0.25,0.63,0.44}{{#1}}}
    \newcommand{\CharTok}[1]{\textcolor[rgb]{0.25,0.44,0.63}{{#1}}}
    \newcommand{\StringTok}[1]{\textcolor[rgb]{0.25,0.44,0.63}{{#1}}}
    \newcommand{\CommentTok}[1]{\textcolor[rgb]{0.38,0.63,0.69}{\textit{{#1}}}}
    \newcommand{\OtherTok}[1]{\textcolor[rgb]{0.00,0.44,0.13}{{#1}}}
    \newcommand{\AlertTok}[1]{\textcolor[rgb]{1.00,0.00,0.00}{\textbf{{#1}}}}
    \newcommand{\FunctionTok}[1]{\textcolor[rgb]{0.02,0.16,0.49}{{#1}}}
    \newcommand{\RegionMarkerTok}[1]{{#1}}
    \newcommand{\ErrorTok}[1]{\textcolor[rgb]{1.00,0.00,0.00}{\textbf{{#1}}}}
    \newcommand{\NormalTok}[1]{{#1}}
    
    % Additional commands for more recent versions of Pandoc
    \newcommand{\ConstantTok}[1]{\textcolor[rgb]{0.53,0.00,0.00}{{#1}}}
    \newcommand{\SpecialCharTok}[1]{\textcolor[rgb]{0.25,0.44,0.63}{{#1}}}
    \newcommand{\VerbatimStringTok}[1]{\textcolor[rgb]{0.25,0.44,0.63}{{#1}}}
    \newcommand{\SpecialStringTok}[1]{\textcolor[rgb]{0.73,0.40,0.53}{{#1}}}
    \newcommand{\ImportTok}[1]{{#1}}
    \newcommand{\DocumentationTok}[1]{\textcolor[rgb]{0.73,0.13,0.13}{\textit{{#1}}}}
    \newcommand{\AnnotationTok}[1]{\textcolor[rgb]{0.38,0.63,0.69}{\textbf{\textit{{#1}}}}}
    \newcommand{\CommentVarTok}[1]{\textcolor[rgb]{0.38,0.63,0.69}{\textbf{\textit{{#1}}}}}
    \newcommand{\VariableTok}[1]{\textcolor[rgb]{0.10,0.09,0.49}{{#1}}}
    \newcommand{\ControlFlowTok}[1]{\textcolor[rgb]{0.00,0.44,0.13}{\textbf{{#1}}}}
    \newcommand{\OperatorTok}[1]{\textcolor[rgb]{0.40,0.40,0.40}{{#1}}}
    \newcommand{\BuiltInTok}[1]{{#1}}
    \newcommand{\ExtensionTok}[1]{{#1}}
    \newcommand{\PreprocessorTok}[1]{\textcolor[rgb]{0.74,0.48,0.00}{{#1}}}
    \newcommand{\AttributeTok}[1]{\textcolor[rgb]{0.49,0.56,0.16}{{#1}}}
    \newcommand{\InformationTok}[1]{\textcolor[rgb]{0.38,0.63,0.69}{\textbf{\textit{{#1}}}}}
    \newcommand{\WarningTok}[1]{\textcolor[rgb]{0.38,0.63,0.69}{\textbf{\textit{{#1}}}}}
    
    
    % Define a nice break command that doesn't care if a line doesn't already
    % exist.
    \def\br{\hspace*{\fill} \\* }
    % Math Jax compatability definitions
    \def\gt{>}
    \def\lt{<}
    % Document parameters
    \title{NTIN084\_-\_HW1\_-\_Jakub\_Mifek}
    
    
    

    % Pygments definitions
    
\makeatletter
\def\PY@reset{\let\PY@it=\relax \let\PY@bf=\relax%
    \let\PY@ul=\relax \let\PY@tc=\relax%
    \let\PY@bc=\relax \let\PY@ff=\relax}
\def\PY@tok#1{\csname PY@tok@#1\endcsname}
\def\PY@toks#1+{\ifx\relax#1\empty\else%
    \PY@tok{#1}\expandafter\PY@toks\fi}
\def\PY@do#1{\PY@bc{\PY@tc{\PY@ul{%
    \PY@it{\PY@bf{\PY@ff{#1}}}}}}}
\def\PY#1#2{\PY@reset\PY@toks#1+\relax+\PY@do{#2}}

\expandafter\def\csname PY@tok@w\endcsname{\def\PY@tc##1{\textcolor[rgb]{0.73,0.73,0.73}{##1}}}
\expandafter\def\csname PY@tok@c\endcsname{\let\PY@it=\textit\def\PY@tc##1{\textcolor[rgb]{0.25,0.50,0.50}{##1}}}
\expandafter\def\csname PY@tok@cp\endcsname{\def\PY@tc##1{\textcolor[rgb]{0.74,0.48,0.00}{##1}}}
\expandafter\def\csname PY@tok@k\endcsname{\let\PY@bf=\textbf\def\PY@tc##1{\textcolor[rgb]{0.00,0.50,0.00}{##1}}}
\expandafter\def\csname PY@tok@kp\endcsname{\def\PY@tc##1{\textcolor[rgb]{0.00,0.50,0.00}{##1}}}
\expandafter\def\csname PY@tok@kt\endcsname{\def\PY@tc##1{\textcolor[rgb]{0.69,0.00,0.25}{##1}}}
\expandafter\def\csname PY@tok@o\endcsname{\def\PY@tc##1{\textcolor[rgb]{0.40,0.40,0.40}{##1}}}
\expandafter\def\csname PY@tok@ow\endcsname{\let\PY@bf=\textbf\def\PY@tc##1{\textcolor[rgb]{0.67,0.13,1.00}{##1}}}
\expandafter\def\csname PY@tok@nb\endcsname{\def\PY@tc##1{\textcolor[rgb]{0.00,0.50,0.00}{##1}}}
\expandafter\def\csname PY@tok@nf\endcsname{\def\PY@tc##1{\textcolor[rgb]{0.00,0.00,1.00}{##1}}}
\expandafter\def\csname PY@tok@nc\endcsname{\let\PY@bf=\textbf\def\PY@tc##1{\textcolor[rgb]{0.00,0.00,1.00}{##1}}}
\expandafter\def\csname PY@tok@nn\endcsname{\let\PY@bf=\textbf\def\PY@tc##1{\textcolor[rgb]{0.00,0.00,1.00}{##1}}}
\expandafter\def\csname PY@tok@ne\endcsname{\let\PY@bf=\textbf\def\PY@tc##1{\textcolor[rgb]{0.82,0.25,0.23}{##1}}}
\expandafter\def\csname PY@tok@nv\endcsname{\def\PY@tc##1{\textcolor[rgb]{0.10,0.09,0.49}{##1}}}
\expandafter\def\csname PY@tok@no\endcsname{\def\PY@tc##1{\textcolor[rgb]{0.53,0.00,0.00}{##1}}}
\expandafter\def\csname PY@tok@nl\endcsname{\def\PY@tc##1{\textcolor[rgb]{0.63,0.63,0.00}{##1}}}
\expandafter\def\csname PY@tok@ni\endcsname{\let\PY@bf=\textbf\def\PY@tc##1{\textcolor[rgb]{0.60,0.60,0.60}{##1}}}
\expandafter\def\csname PY@tok@na\endcsname{\def\PY@tc##1{\textcolor[rgb]{0.49,0.56,0.16}{##1}}}
\expandafter\def\csname PY@tok@nt\endcsname{\let\PY@bf=\textbf\def\PY@tc##1{\textcolor[rgb]{0.00,0.50,0.00}{##1}}}
\expandafter\def\csname PY@tok@nd\endcsname{\def\PY@tc##1{\textcolor[rgb]{0.67,0.13,1.00}{##1}}}
\expandafter\def\csname PY@tok@s\endcsname{\def\PY@tc##1{\textcolor[rgb]{0.73,0.13,0.13}{##1}}}
\expandafter\def\csname PY@tok@sd\endcsname{\let\PY@it=\textit\def\PY@tc##1{\textcolor[rgb]{0.73,0.13,0.13}{##1}}}
\expandafter\def\csname PY@tok@si\endcsname{\let\PY@bf=\textbf\def\PY@tc##1{\textcolor[rgb]{0.73,0.40,0.53}{##1}}}
\expandafter\def\csname PY@tok@se\endcsname{\let\PY@bf=\textbf\def\PY@tc##1{\textcolor[rgb]{0.73,0.40,0.13}{##1}}}
\expandafter\def\csname PY@tok@sr\endcsname{\def\PY@tc##1{\textcolor[rgb]{0.73,0.40,0.53}{##1}}}
\expandafter\def\csname PY@tok@ss\endcsname{\def\PY@tc##1{\textcolor[rgb]{0.10,0.09,0.49}{##1}}}
\expandafter\def\csname PY@tok@sx\endcsname{\def\PY@tc##1{\textcolor[rgb]{0.00,0.50,0.00}{##1}}}
\expandafter\def\csname PY@tok@m\endcsname{\def\PY@tc##1{\textcolor[rgb]{0.40,0.40,0.40}{##1}}}
\expandafter\def\csname PY@tok@gh\endcsname{\let\PY@bf=\textbf\def\PY@tc##1{\textcolor[rgb]{0.00,0.00,0.50}{##1}}}
\expandafter\def\csname PY@tok@gu\endcsname{\let\PY@bf=\textbf\def\PY@tc##1{\textcolor[rgb]{0.50,0.00,0.50}{##1}}}
\expandafter\def\csname PY@tok@gd\endcsname{\def\PY@tc##1{\textcolor[rgb]{0.63,0.00,0.00}{##1}}}
\expandafter\def\csname PY@tok@gi\endcsname{\def\PY@tc##1{\textcolor[rgb]{0.00,0.63,0.00}{##1}}}
\expandafter\def\csname PY@tok@gr\endcsname{\def\PY@tc##1{\textcolor[rgb]{1.00,0.00,0.00}{##1}}}
\expandafter\def\csname PY@tok@ge\endcsname{\let\PY@it=\textit}
\expandafter\def\csname PY@tok@gs\endcsname{\let\PY@bf=\textbf}
\expandafter\def\csname PY@tok@gp\endcsname{\let\PY@bf=\textbf\def\PY@tc##1{\textcolor[rgb]{0.00,0.00,0.50}{##1}}}
\expandafter\def\csname PY@tok@go\endcsname{\def\PY@tc##1{\textcolor[rgb]{0.53,0.53,0.53}{##1}}}
\expandafter\def\csname PY@tok@gt\endcsname{\def\PY@tc##1{\textcolor[rgb]{0.00,0.27,0.87}{##1}}}
\expandafter\def\csname PY@tok@err\endcsname{\def\PY@bc##1{\setlength{\fboxsep}{0pt}\fcolorbox[rgb]{1.00,0.00,0.00}{1,1,1}{\strut ##1}}}
\expandafter\def\csname PY@tok@kc\endcsname{\let\PY@bf=\textbf\def\PY@tc##1{\textcolor[rgb]{0.00,0.50,0.00}{##1}}}
\expandafter\def\csname PY@tok@kd\endcsname{\let\PY@bf=\textbf\def\PY@tc##1{\textcolor[rgb]{0.00,0.50,0.00}{##1}}}
\expandafter\def\csname PY@tok@kn\endcsname{\let\PY@bf=\textbf\def\PY@tc##1{\textcolor[rgb]{0.00,0.50,0.00}{##1}}}
\expandafter\def\csname PY@tok@kr\endcsname{\let\PY@bf=\textbf\def\PY@tc##1{\textcolor[rgb]{0.00,0.50,0.00}{##1}}}
\expandafter\def\csname PY@tok@bp\endcsname{\def\PY@tc##1{\textcolor[rgb]{0.00,0.50,0.00}{##1}}}
\expandafter\def\csname PY@tok@fm\endcsname{\def\PY@tc##1{\textcolor[rgb]{0.00,0.00,1.00}{##1}}}
\expandafter\def\csname PY@tok@vc\endcsname{\def\PY@tc##1{\textcolor[rgb]{0.10,0.09,0.49}{##1}}}
\expandafter\def\csname PY@tok@vg\endcsname{\def\PY@tc##1{\textcolor[rgb]{0.10,0.09,0.49}{##1}}}
\expandafter\def\csname PY@tok@vi\endcsname{\def\PY@tc##1{\textcolor[rgb]{0.10,0.09,0.49}{##1}}}
\expandafter\def\csname PY@tok@vm\endcsname{\def\PY@tc##1{\textcolor[rgb]{0.10,0.09,0.49}{##1}}}
\expandafter\def\csname PY@tok@sa\endcsname{\def\PY@tc##1{\textcolor[rgb]{0.73,0.13,0.13}{##1}}}
\expandafter\def\csname PY@tok@sb\endcsname{\def\PY@tc##1{\textcolor[rgb]{0.73,0.13,0.13}{##1}}}
\expandafter\def\csname PY@tok@sc\endcsname{\def\PY@tc##1{\textcolor[rgb]{0.73,0.13,0.13}{##1}}}
\expandafter\def\csname PY@tok@dl\endcsname{\def\PY@tc##1{\textcolor[rgb]{0.73,0.13,0.13}{##1}}}
\expandafter\def\csname PY@tok@s2\endcsname{\def\PY@tc##1{\textcolor[rgb]{0.73,0.13,0.13}{##1}}}
\expandafter\def\csname PY@tok@sh\endcsname{\def\PY@tc##1{\textcolor[rgb]{0.73,0.13,0.13}{##1}}}
\expandafter\def\csname PY@tok@s1\endcsname{\def\PY@tc##1{\textcolor[rgb]{0.73,0.13,0.13}{##1}}}
\expandafter\def\csname PY@tok@mb\endcsname{\def\PY@tc##1{\textcolor[rgb]{0.40,0.40,0.40}{##1}}}
\expandafter\def\csname PY@tok@mf\endcsname{\def\PY@tc##1{\textcolor[rgb]{0.40,0.40,0.40}{##1}}}
\expandafter\def\csname PY@tok@mh\endcsname{\def\PY@tc##1{\textcolor[rgb]{0.40,0.40,0.40}{##1}}}
\expandafter\def\csname PY@tok@mi\endcsname{\def\PY@tc##1{\textcolor[rgb]{0.40,0.40,0.40}{##1}}}
\expandafter\def\csname PY@tok@il\endcsname{\def\PY@tc##1{\textcolor[rgb]{0.40,0.40,0.40}{##1}}}
\expandafter\def\csname PY@tok@mo\endcsname{\def\PY@tc##1{\textcolor[rgb]{0.40,0.40,0.40}{##1}}}
\expandafter\def\csname PY@tok@ch\endcsname{\let\PY@it=\textit\def\PY@tc##1{\textcolor[rgb]{0.25,0.50,0.50}{##1}}}
\expandafter\def\csname PY@tok@cm\endcsname{\let\PY@it=\textit\def\PY@tc##1{\textcolor[rgb]{0.25,0.50,0.50}{##1}}}
\expandafter\def\csname PY@tok@cpf\endcsname{\let\PY@it=\textit\def\PY@tc##1{\textcolor[rgb]{0.25,0.50,0.50}{##1}}}
\expandafter\def\csname PY@tok@c1\endcsname{\let\PY@it=\textit\def\PY@tc##1{\textcolor[rgb]{0.25,0.50,0.50}{##1}}}
\expandafter\def\csname PY@tok@cs\endcsname{\let\PY@it=\textit\def\PY@tc##1{\textcolor[rgb]{0.25,0.50,0.50}{##1}}}

\def\PYZbs{\char`\\}
\def\PYZus{\char`\_}
\def\PYZob{\char`\{}
\def\PYZcb{\char`\}}
\def\PYZca{\char`\^}
\def\PYZam{\char`\&}
\def\PYZlt{\char`\<}
\def\PYZgt{\char`\>}
\def\PYZsh{\char`\#}
\def\PYZpc{\char`\%}
\def\PYZdl{\char`\$}
\def\PYZhy{\char`\-}
\def\PYZsq{\char`\'}
\def\PYZdq{\char`\"}
\def\PYZti{\char`\~}
% for compatibility with earlier versions
\def\PYZat{@}
\def\PYZlb{[}
\def\PYZrb{]}
\makeatother


    % Exact colors from NB
    \definecolor{incolor}{rgb}{0.0, 0.0, 0.5}
    \definecolor{outcolor}{rgb}{0.545, 0.0, 0.0}



    
    % Prevent overflowing lines due to hard-to-break entities
    \sloppy 
    % Setup hyperref package
    \hypersetup{
      breaklinks=true,  % so long urls are correctly broken across lines
      colorlinks=true,
      urlcolor=urlcolor,
      linkcolor=linkcolor,
      citecolor=citecolor,
      }
    % Slightly bigger margins than the latex defaults
    
    \geometry{verbose,tmargin=1in,bmargin=1in,lmargin=1in,rmargin=1in}
    
    

    \begin{document}
    
    
    \maketitle
    
    

    
    \section{Úloha 1 - určovanie príbuznosti pomocou
kompresie}\label{uxfaloha-1---urux10dovanie-pruxedbuznosti-pomocou-kompresie}

    \begin{Verbatim}[commandchars=\\\{\}]
{\color{incolor}In [{\color{incolor}1}]:} \PY{k+kn}{import} \PY{n+nn}{gzip}
        
        \PY{k}{def} \PY{n+nf}{loadfasta}\PY{p}{(}\PY{n}{filename}\PY{p}{,}\PY{n}{verbose}\PY{o}{=}\PY{l+m+mi}{0}\PY{p}{)}\PY{p}{:}
            \PY{l+s+sd}{\PYZdq{}\PYZdq{}\PYZdq{} Parses a classically formatted and possibly }
        \PY{l+s+sd}{        compressed FASTA file into a dictionary where the key}
        \PY{l+s+sd}{        for a sequence is the first part of its header without }
        \PY{l+s+sd}{        any white space; if verbose is nonzero then the identifiers }
        \PY{l+s+sd}{        together with lengths of the read sequences are printed\PYZdq{}\PYZdq{}\PYZdq{}}
            \PY{k}{if} \PY{p}{(}\PY{n}{filename}\PY{o}{.}\PY{n}{endswith}\PY{p}{(}\PY{l+s+s2}{\PYZdq{}}\PY{l+s+s2}{.gz}\PY{l+s+s2}{\PYZdq{}}\PY{p}{)}\PY{p}{)}\PY{p}{:}
                \PY{n}{fp} \PY{o}{=} \PY{n}{gzip}\PY{o}{.}\PY{n}{open}\PY{p}{(}\PY{n}{filename}\PY{p}{,} \PY{l+s+s1}{\PYZsq{}}\PY{l+s+s1}{rt}\PY{l+s+s1}{\PYZsq{}}\PY{p}{)}
            \PY{k}{else}\PY{p}{:}
                \PY{n}{fp} \PY{o}{=} \PY{n+nb}{open}\PY{p}{(}\PY{n}{filename}\PY{p}{,} \PY{l+s+s1}{\PYZsq{}}\PY{l+s+s1}{r}\PY{l+s+s1}{\PYZsq{}}\PY{p}{)}
            \PY{c+c1}{\PYZsh{} split at headers}
            \PY{c+c1}{\PYZsh{} data = fp.read().split(\PYZsq{}\PYZgt{}\PYZsq{})}
            \PY{n}{data} \PY{o}{=} \PY{n}{fp}\PY{o}{.}\PY{n}{read}\PY{p}{(}\PY{p}{)}
            \PY{n}{data} \PY{o}{=} \PY{n}{data}\PY{o}{.}\PY{n}{split}\PY{p}{(}\PY{l+s+s1}{\PYZsq{}}\PY{l+s+s1}{\PYZgt{}}\PY{l+s+s1}{\PYZsq{}}\PY{p}{)}
            \PY{n}{fp}\PY{o}{.}\PY{n}{close}\PY{p}{(}\PY{p}{)}
            \PY{c+c1}{\PYZsh{} ignore whatever appears before the 1st header}
            \PY{n}{data}\PY{o}{.}\PY{n}{pop}\PY{p}{(}\PY{l+m+mi}{0}\PY{p}{)}     
            \PY{c+c1}{\PYZsh{} prepare the dictionary}
            \PY{n}{D} \PY{o}{=} \PY{p}{\PYZob{}}\PY{p}{\PYZcb{}}
            \PY{k}{for} \PY{n}{sequence} \PY{o+ow}{in} \PY{n}{data}\PY{p}{:}
                \PY{n}{lines} \PY{o}{=} \PY{n}{sequence}\PY{o}{.}\PY{n}{split}\PY{p}{(}\PY{l+s+s1}{\PYZsq{}}\PY{l+s+se}{\PYZbs{}n}\PY{l+s+s1}{\PYZsq{}}\PY{p}{)}
                \PY{n}{header} \PY{o}{=} \PY{n}{lines}\PY{o}{.}\PY{n}{pop}\PY{p}{(}\PY{l+m+mi}{0}\PY{p}{)}\PY{o}{.}\PY{n}{split}\PY{p}{(}\PY{p}{)}
                \PY{n}{key} \PY{o}{=} \PY{n}{header}\PY{p}{[}\PY{l+m+mi}{0}\PY{p}{]}
                \PY{n}{D}\PY{p}{[}\PY{n}{key}\PY{p}{]} \PY{o}{=} \PY{l+s+s1}{\PYZsq{}}\PY{l+s+s1}{\PYZsq{}}\PY{o}{.}\PY{n}{join}\PY{p}{(}\PY{n}{lines}\PY{p}{)}
                \PY{k}{if} \PY{n}{verbose}\PY{p}{:}
                    \PY{n+nb}{print}\PY{p}{(}\PY{l+s+s2}{\PYZdq{}}\PY{l+s+s2}{Sequence }\PY{l+s+si}{\PYZpc{}s}\PY{l+s+s2}{ of length }\PY{l+s+si}{\PYZpc{}d}\PY{l+s+s2}{ read}\PY{l+s+s2}{\PYZdq{}} \PY{o}{\PYZpc{}} \PY{p}{(}\PY{n}{key}\PY{p}{,}\PY{n+nb}{len}\PY{p}{(}\PY{n}{D}\PY{p}{[}\PY{n}{key}\PY{p}{]}\PY{p}{)}\PY{p}{)}\PY{p}{)}
            \PY{k}{return} \PY{n}{D}
        
        \PY{n}{seq} \PY{o}{=} \PY{n}{loadfasta}\PY{p}{(}\PY{l+s+s1}{\PYZsq{}}\PY{l+s+s1}{Seq.fasta}\PY{l+s+s1}{\PYZsq{}}\PY{p}{)}
\end{Verbatim}


    First we read \texttt{Seq.fasta} file. We do that using the
\texttt{loadfasta} method from seminar.

    \begin{Verbatim}[commandchars=\\\{\}]
{\color{incolor}In [{\color{incolor}2}]:} \PY{k+kn}{from} \PY{n+nn}{gzip} \PY{k}{import} \PY{n}{compress} \PY{k}{as} \PY{n}{gzip\PYZus{}compress}
        \PY{k+kn}{from} \PY{n+nn}{os}\PY{n+nn}{.}\PY{n+nn}{path} \PY{k}{import} \PY{n}{getsize}\PY{p}{,} \PY{n}{splitext}
        \PY{k+kn}{from} \PY{n+nn}{subprocess} \PY{k}{import} \PY{n}{run}
        
        \PY{k}{def} \PY{n+nf}{gencompress\PYZus{}size}\PY{p}{(}\PY{n}{genom}\PY{p}{,} \PY{n}{reference}\PY{o}{=}\PY{k+kc}{None}\PY{p}{)}\PY{p}{:}
            \PY{n}{gename} \PY{o}{=} \PY{l+s+s2}{\PYZdq{}}\PY{l+s+s2}{gene}\PY{l+s+s2}{\PYZdq{}}
            \PY{n}{genomfile} \PY{o}{=} \PY{n}{gename}\PY{o}{+}\PY{l+s+s1}{\PYZsq{}}\PY{l+s+s1}{.tmp}\PY{l+s+s1}{\PYZsq{}}
            
            \PY{k}{with} \PY{n+nb}{open}\PY{p}{(}\PY{n}{genomfile}\PY{p}{,} \PY{l+s+s2}{\PYZdq{}}\PY{l+s+s2}{w+}\PY{l+s+s2}{\PYZdq{}}\PY{p}{)} \PY{k}{as} \PY{n}{file}\PY{p}{:}
                \PY{n}{file}\PY{o}{.}\PY{n}{write}\PY{p}{(}\PY{n}{genom}\PY{p}{)}
            
            \PY{k}{if} \PY{o+ow}{not} \PY{n}{reference} \PY{o}{==} \PY{k+kc}{None}\PY{p}{:}
                \PY{n}{refname} \PY{o}{=} \PY{l+s+s2}{\PYZdq{}}\PY{l+s+s2}{ref}\PY{l+s+s2}{\PYZdq{}}
                \PY{n}{refile} \PY{o}{=} \PY{n}{refname}\PY{o}{+}\PY{l+s+s1}{\PYZsq{}}\PY{l+s+s1}{.tmp}\PY{l+s+s1}{\PYZsq{}}
            
                \PY{k}{with} \PY{n+nb}{open}\PY{p}{(}\PY{n}{refile}\PY{p}{,} \PY{l+s+s2}{\PYZdq{}}\PY{l+s+s2}{w+}\PY{l+s+s2}{\PYZdq{}}\PY{p}{)} \PY{k}{as} \PY{n}{file}\PY{p}{:}
                    \PY{n}{file}\PY{o}{.}\PY{n}{write}\PY{p}{(}\PY{n}{reference}\PY{p}{)}
                    
                \PY{n}{run}\PY{p}{(}\PY{p}{[}\PY{l+s+s2}{\PYZdq{}}\PY{l+s+s2}{GenCompress.exe}\PY{l+s+s2}{\PYZdq{}}\PY{p}{,} \PY{n}{genomfile}\PY{p}{,} \PY{l+s+s2}{\PYZdq{}}\PY{l+s+s2}{\PYZhy{}c}\PY{l+s+s2}{\PYZdq{}}\PY{p}{,} \PY{n}{refile}\PY{p}{]}\PY{p}{)}
            \PY{k}{else}\PY{p}{:}
                \PY{n}{run}\PY{p}{(}\PY{p}{[}\PY{l+s+s2}{\PYZdq{}}\PY{l+s+s2}{GenCompress.exe}\PY{l+s+s2}{\PYZdq{}}\PY{p}{,} \PY{n}{genomfile}\PY{p}{]}\PY{p}{)}
             
            \PY{k}{return} \PY{n}{getsize}\PY{p}{(}\PY{n}{gename}\PY{o}{+}\PY{l+s+s2}{\PYZdq{}}\PY{l+s+s2}{.GEN}\PY{l+s+s2}{\PYZdq{}}\PY{p}{)}
        
        \PY{k}{def} \PY{n+nf}{gzipcompress\PYZus{}size}\PY{p}{(}\PY{n}{genom}\PY{p}{,} \PY{n}{reference}\PY{o}{=}\PY{k+kc}{None}\PY{p}{)}\PY{p}{:}
            \PY{n}{size} \PY{o}{=} \PY{n+nb}{len}\PY{p}{(}\PY{n}{gzip\PYZus{}compress}\PY{p}{(}\PY{n+nb}{bytes}\PY{p}{(}\PY{n}{genom}\PY{p}{,} \PY{n}{encoding}\PY{o}{=}\PY{l+s+s1}{\PYZsq{}}\PY{l+s+s1}{utf8}\PY{l+s+s1}{\PYZsq{}}\PY{p}{)}\PY{p}{)}\PY{p}{)}
            
            \PY{k}{if} \PY{o+ow}{not} \PY{n}{reference} \PY{o}{==} \PY{k+kc}{None}\PY{p}{:}
                \PY{n}{size} \PY{o}{=} \PY{n+nb}{len}\PY{p}{(}\PY{n}{gzip\PYZus{}compress}\PY{p}{(}\PY{n+nb}{bytes}\PY{p}{(}\PY{n}{reference} \PY{o}{+} \PY{n}{genom}\PY{p}{,} \PY{n}{encoding}\PY{o}{=}\PY{l+s+s1}{\PYZsq{}}\PY{l+s+s1}{utf8}\PY{l+s+s1}{\PYZsq{}}\PY{p}{)}\PY{p}{)}\PY{p}{)} \PY{o}{\PYZhy{}} \PY{n}{size}
            \PY{k}{return} \PY{n}{size}
        
        \PY{n}{compressize} \PY{o}{=} \PY{p}{\PYZob{}}
            \PY{l+s+s2}{\PYZdq{}}\PY{l+s+s2}{gen}\PY{l+s+s2}{\PYZdq{}}\PY{p}{:} \PY{n}{gencompress\PYZus{}size}\PY{p}{,}
            \PY{l+s+s2}{\PYZdq{}}\PY{l+s+s2}{gzip}\PY{l+s+s2}{\PYZdq{}}\PY{p}{:} \PY{n}{gzipcompress\PYZus{}size}
        \PY{p}{\PYZcb{}}
\end{Verbatim}


    Then we define methods for computing size of compressed sequences with
or without reference sequences using different algorithms
(\textbf{GenCompress} and \textbf{gzip}). Beside the methods we define
compressize switch variable for more transparent use.

    \begin{Verbatim}[commandchars=\\\{\}]
{\color{incolor}In [{\color{incolor}3}]:} \PY{n}{compressize}\PY{p}{[}\PY{l+s+s1}{\PYZsq{}}\PY{l+s+s1}{gen}\PY{l+s+s1}{\PYZsq{}}\PY{p}{]}\PY{p}{(}\PY{n}{seq}\PY{p}{[}\PY{l+s+s1}{\PYZsq{}}\PY{l+s+s1}{A}\PY{l+s+s1}{\PYZsq{}}\PY{p}{]}\PY{p}{,} \PY{n}{seq}\PY{p}{[}\PY{l+s+s1}{\PYZsq{}}\PY{l+s+s1}{B}\PY{l+s+s1}{\PYZsq{}}\PY{p}{]}\PY{p}{)}
\end{Verbatim}


\begin{Verbatim}[commandchars=\\\{\}]
{\color{outcolor}Out[{\color{outcolor}3}]:} 1377
\end{Verbatim}
            
    \begin{Verbatim}[commandchars=\\\{\}]
{\color{incolor}In [{\color{incolor}4}]:} \PY{n}{compressize}\PY{p}{[}\PY{l+s+s1}{\PYZsq{}}\PY{l+s+s1}{gzip}\PY{l+s+s1}{\PYZsq{}}\PY{p}{]}\PY{p}{(}\PY{n}{seq}\PY{p}{[}\PY{l+s+s1}{\PYZsq{}}\PY{l+s+s1}{A}\PY{l+s+s1}{\PYZsq{}}\PY{p}{]}\PY{p}{,} \PY{n}{seq}\PY{p}{[}\PY{l+s+s1}{\PYZsq{}}\PY{l+s+s1}{B}\PY{l+s+s1}{\PYZsq{}}\PY{p}{]}\PY{p}{)}
\end{Verbatim}


\begin{Verbatim}[commandchars=\\\{\}]
{\color{outcolor}Out[{\color{outcolor}4}]:} 1354
\end{Verbatim}
            
    Now we need to define distances. The distance will be computed using the
formula from the assignment and expressed in percentage rounded to 0
digits.

    \begin{Verbatim}[commandchars=\\\{\}]
{\color{incolor}In [{\color{incolor}5}]:} \PY{k}{def} \PY{n+nf}{distance}\PY{p}{(}\PY{n}{genA}\PY{p}{,} \PY{n}{genB}\PY{p}{,} \PY{n}{compress} \PY{o}{=} \PY{n}{compressize}\PY{p}{[}\PY{l+s+s1}{\PYZsq{}}\PY{l+s+s1}{gen}\PY{l+s+s1}{\PYZsq{}}\PY{p}{]}\PY{p}{)}\PY{p}{:}
            \PY{k}{return} \PY{n+nb}{round}\PY{p}{(}\PY{p}{(}\PY{l+m+mi}{1} \PY{o}{\PYZhy{}} \PY{p}{(}\PY{n}{compress}\PY{p}{(}\PY{n}{genA}\PY{p}{)} \PY{o}{\PYZhy{}} \PY{n}{compress}\PY{p}{(}\PY{n}{genA}\PY{p}{,} \PY{n}{genB}\PY{p}{)}\PY{p}{)} \PY{o}{/} \PY{n}{compress}\PY{p}{(}\PY{n}{genA} \PY{o}{+} \PY{n}{genB}\PY{p}{)}\PY{p}{)} \PY{o}{*} \PY{l+m+mi}{100}\PY{p}{)}
\end{Verbatim}


    \begin{Verbatim}[commandchars=\\\{\}]
{\color{incolor}In [{\color{incolor}6}]:} \PY{n}{distance}\PY{p}{(}\PY{n}{seq}\PY{p}{[}\PY{l+s+s1}{\PYZsq{}}\PY{l+s+s1}{A}\PY{l+s+s1}{\PYZsq{}}\PY{p}{]}\PY{p}{,} \PY{n}{seq}\PY{p}{[}\PY{l+s+s1}{\PYZsq{}}\PY{l+s+s1}{B}\PY{l+s+s1}{\PYZsq{}}\PY{p}{]}\PY{p}{,} \PY{n}{compressize}\PY{p}{[}\PY{l+s+s1}{\PYZsq{}}\PY{l+s+s1}{gen}\PY{l+s+s1}{\PYZsq{}}\PY{p}{]}\PY{p}{)}
\end{Verbatim}


\begin{Verbatim}[commandchars=\\\{\}]
{\color{outcolor}Out[{\color{outcolor}6}]:} 95
\end{Verbatim}
            
    Distance table computation:

    \begin{Verbatim}[commandchars=\\\{\}]
{\color{incolor}In [{\color{incolor}7}]:} \PY{k+kn}{from} \PY{n+nn}{math} \PY{k}{import} \PY{n}{inf}
        
        \PY{k}{def} \PY{n+nf}{distance\PYZus{}table}\PY{p}{(}\PY{n}{sequencedict}\PY{p}{,} \PY{n}{compress} \PY{o}{=} \PY{n}{compressize}\PY{p}{[}\PY{l+s+s1}{\PYZsq{}}\PY{l+s+s1}{gen}\PY{l+s+s1}{\PYZsq{}}\PY{p}{]}\PY{p}{)}\PY{p}{:}
            \PY{n}{table} \PY{o}{=} \PY{p}{[}\PY{p}{[}\PY{n}{inf} \PY{k}{for} \PY{n}{x} \PY{o+ow}{in} \PY{n+nb}{range}\PY{p}{(}\PY{n+nb}{len}\PY{p}{(}\PY{n}{sequencedict}\PY{p}{)}\PY{p}{)}\PY{p}{]} \PY{k}{for} \PY{n}{y} \PY{o+ow}{in} \PY{n+nb}{range}\PY{p}{(}\PY{n+nb}{len}\PY{p}{(}\PY{n}{sequencedict}\PY{p}{)}\PY{p}{)}\PY{p}{]}
            \PY{n}{names} \PY{o}{=} \PY{p}{[}\PY{n}{key} \PY{k}{for} \PY{n}{key} \PY{o+ow}{in} \PY{n}{sequencedict}\PY{o}{.}\PY{n}{keys}\PY{p}{(}\PY{p}{)}\PY{p}{]}
            \PY{k}{for} \PY{n}{A} \PY{o+ow}{in} \PY{n+nb}{range}\PY{p}{(}\PY{n+nb}{len}\PY{p}{(}\PY{n}{names}\PY{p}{)}\PY{p}{)}\PY{p}{:}
                \PY{k}{for} \PY{n}{B} \PY{o+ow}{in} \PY{n+nb}{range}\PY{p}{(}\PY{n+nb}{len}\PY{p}{(}\PY{n}{names}\PY{p}{)}\PY{p}{)}\PY{p}{:}
                    \PY{k}{if} \PY{n}{A} \PY{o}{\PYZlt{}} \PY{n}{B}\PY{p}{:}
                        \PY{n}{table}\PY{p}{[}\PY{n}{A}\PY{p}{]}\PY{p}{[}\PY{n}{B}\PY{p}{]} \PY{o}{=} \PY{n}{distance}\PY{p}{(}\PY{n}{sequencedict}\PY{p}{[}\PY{n}{names}\PY{p}{[}\PY{n}{A}\PY{p}{]}\PY{p}{]}\PY{p}{,} \PY{n}{sequencedict}\PY{p}{[}\PY{n}{names}\PY{p}{[}\PY{n}{B}\PY{p}{]}\PY{p}{]}\PY{p}{,} \PY{n}{compress}\PY{p}{)}
                    \PY{k}{elif} \PY{n}{A} \PY{o}{==} \PY{n}{B}\PY{p}{:}
                        \PY{n}{table}\PY{p}{[}\PY{n}{A}\PY{p}{]}\PY{p}{[}\PY{n}{B}\PY{p}{]} \PY{o}{=} \PY{l+m+mi}{0}
                    \PY{k}{else}\PY{p}{:}
                        \PY{n}{table}\PY{p}{[}\PY{n}{A}\PY{p}{]}\PY{p}{[}\PY{n}{B}\PY{p}{]} \PY{o}{=} \PY{n}{table}\PY{p}{[}\PY{n}{B}\PY{p}{]}\PY{p}{[}\PY{n}{A}\PY{p}{]}
            
            \PY{k}{return} \PY{n}{names}\PY{p}{,} \PY{n}{table}
\end{Verbatim}


    \begin{Verbatim}[commandchars=\\\{\}]
{\color{incolor}In [{\color{incolor}8}]:} \PY{n}{distance\PYZus{}table}\PY{p}{(}\PY{n}{seq}\PY{p}{,} \PY{n}{compressize}\PY{p}{[}\PY{l+s+s1}{\PYZsq{}}\PY{l+s+s1}{gzip}\PY{l+s+s1}{\PYZsq{}}\PY{p}{]}\PY{p}{)}
\end{Verbatim}


\begin{Verbatim}[commandchars=\\\{\}]
{\color{outcolor}Out[{\color{outcolor}8}]:} (['A', 'B', 'C', 'D', 'E', 'F', 'G', 'H'],
         [[0, 83, 96, 81, 74, 81, 68, 62],
          [83, 0, 90, 50, 62, 46, 72, 66],
          [96, 90, 0, 60, 50, 58, 62, 57],
          [81, 50, 60, 0, 67, 45, 75, 68],
          [74, 62, 50, 67, 0, 80, 80, 74],
          [81, 46, 58, 45, 80, 0, 75, 68],
          [68, 72, 62, 75, 80, 75, 0, 79],
          [62, 66, 57, 68, 74, 68, 79, 0]])
\end{Verbatim}
            
    Compute both tables:

    \begin{Verbatim}[commandchars=\\\{\}]
{\color{incolor}In [{\color{incolor}9}]:} \PY{n}{tables} \PY{o}{=} \PY{p}{\PYZob{}}\PY{p}{\PYZcb{}}
        \PY{k}{for} \PY{n}{compression} \PY{o+ow}{in} \PY{n}{compressize}\PY{p}{:}
            \PY{n+nb}{print}\PY{p}{(}\PY{l+s+s1}{\PYZsq{}}\PY{l+s+s1}{Computing distance table using compression:}\PY{l+s+s1}{\PYZsq{}}\PY{p}{,} \PY{n}{compression}\PY{p}{)}
            \PY{n}{tables}\PY{p}{[}\PY{n}{compression}\PY{p}{]} \PY{o}{=} \PY{n}{distance\PYZus{}table}\PY{p}{(}\PY{n}{seq}\PY{p}{,} \PY{n}{compressize}\PY{p}{[}\PY{n}{compression}\PY{p}{]}\PY{p}{)}
\end{Verbatim}


    \begin{Verbatim}[commandchars=\\\{\}]
Computing distance table using compression: gen
Computing distance table using compression: gzip

    \end{Verbatim}

    \begin{Verbatim}[commandchars=\\\{\}]
{\color{incolor}In [{\color{incolor}10}]:} \PY{k+kn}{from} \PY{n+nn}{pprint} \PY{k}{import} \PY{n}{pprint}
         \PY{n}{pprint}\PY{p}{(}\PY{n}{tables}\PY{p}{)}
\end{Verbatim}


    \begin{Verbatim}[commandchars=\\\{\}]
\{'gen': (['A', 'B', 'C', 'D', 'E', 'F', 'G', 'H'],
         [[0, 95, 96, 95, 95, 95, 96, 97],
          [95, 0, 73, 50, 69, 49, 91, 93],
          [96, 73, 0, 76, 72, 76, 93, 94],
          [95, 50, 76, 0, 71, 44, 91, 92],
          [95, 69, 72, 71, 0, 71, 90, 92],
          [95, 49, 76, 44, 71, 0, 91, 93],
          [96, 91, 93, 91, 90, 91, 0, 90],
          [97, 93, 94, 92, 92, 93, 90, 0]]),
 'gzip': (['A', 'B', 'C', 'D', 'E', 'F', 'G', 'H'],
          [[0, 83, 96, 81, 74, 81, 68, 62],
           [83, 0, 90, 50, 62, 46, 72, 66],
           [96, 90, 0, 60, 50, 58, 62, 57],
           [81, 50, 60, 0, 67, 45, 75, 68],
           [74, 62, 50, 67, 0, 80, 80, 74],
           [81, 46, 58, 45, 80, 0, 75, 68],
           [68, 72, 62, 75, 80, 75, 0, 79],
           [62, 66, 57, 68, 74, 68, 79, 0]])\}

    \end{Verbatim}

    From the tables we just generated we can see the probable affinity of
species.

For \textbf{gen}: 1. F and D (44) 2. B and F (49) 3. B and D (50)

Thus, we can conclude that \emph{F}, \emph{D} and \emph{B} are very
close to each other.

For \textbf{gzip}: 1. F and D (45) 2. F and B (46) 3. B and D (50)

Thus, we can see that \emph{F}, \emph{D} and \emph{B} are very close in
this compression as well.

From this we can conclude that in both polygenetic trees nodes
representing these species will look similar to:

\begin{verbatim}
   /--B
--|    /--F
   \--|
       \--D
\end{verbatim}

The result is not same in all cases though. For example the last row
denoting the affinity of \emph{H} to other species differs a lot. For
\textbf{gen} the compression denotes almost none affinity to any of the
other species. For \textbf{gzip}, however, there is very high affinity
to \emph{C} - 57 (for \textbf{gen} the affinity to \emph{C} was 94).

From this we can conclude that the tree will differ for \emph{H} - for
\textbf{gzip} tree there should be visible affinity to \emph{C} and for
\textbf{gen} the \emph{H} node should be somewhere at the top layers of
the tree, with high distances to other nodes (closest should be \emph{G}
with distance 90).

As of which algorithm is more suitable for polygenetic tree creation, I
would say that \textbf{GenCompress} suits better the need. Our aim is
not to compress genomes as much as possible but to find similarities in
genoms. With \textbf{GenCompress} we actually try to compress one genom
using the other. With \textbf{gzip} we just concatenate both sequences
and hope that it will use the first for referencing the other. But that
does not have to be the case and the compression algorithm may have used
some other method to obtain the best result and therefore we cannot
\emph{trust} the \textbf{gzip} result as much as the
\textbf{GenCompress} result.

    Now we need to create the phylogenetic tree. We will use \texttt{ete3}
library designed for phylogenetic tree representation. For more
information about the library see
\href{http://etetoolkit.org/ipython_notebook/}{ete3}.

Before we can create the tree, we need to cluster the data using the
distance matrix. For this purpose I use
\href{https://pypi.org/project/dedupe-hcluster/}{dedupe-hcluster}
module.

    \begin{Verbatim}[commandchars=\\\{\}]
{\color{incolor}In [{\color{incolor}22}]:} \PY{k+kn}{import} \PY{n+nn}{sys}
         \PY{o}{!}\PY{o}{\PYZob{}}sys.executable\PY{o}{\PYZcb{}} \PYZhy{}m pip install dedupe\PYZhy{}hcluster
         \PY{o}{!}\PY{o}{\PYZob{}}sys.executable\PY{o}{\PYZcb{}} \PYZhy{}m pip install ete3
\end{Verbatim}


    \begin{Verbatim}[commandchars=\\\{\}]
Requirement already satisfied: dedupe-hcluster in c:\textbackslash{}users\textbackslash{}jakub\textbackslash{}anaconda3\textbackslash{}lib\textbackslash{}site-packages (0.3.6)
Requirement already satisfied: numpy>=1.12.1; python\_version == "3.6" in c:\textbackslash{}users\textbackslash{}jakub\textbackslash{}anaconda3\textbackslash{}lib\textbackslash{}site-packages (from dedupe-hcluster) (1.14.3)
Requirement already satisfied: future in c:\textbackslash{}users\textbackslash{}jakub\textbackslash{}anaconda3\textbackslash{}lib\textbackslash{}site-packages (from dedupe-hcluster) (0.17.1)

    \end{Verbatim}

    \begin{Verbatim}[commandchars=\\\{\}]
You are using pip version 18.0, however version 18.1 is available.
You should consider upgrading via the 'python -m pip install --upgrade pip' command.

    \end{Verbatim}

    \begin{Verbatim}[commandchars=\\\{\}]
Requirement already satisfied: ete3 in c:\textbackslash{}users\textbackslash{}jakub\textbackslash{}anaconda3\textbackslash{}lib\textbackslash{}site-packages (3.1.1)

    \end{Verbatim}

    \begin{Verbatim}[commandchars=\\\{\}]
You are using pip version 18.0, however version 18.1 is available.
You should consider upgrading via the 'python -m pip install --upgrade pip' command.

    \end{Verbatim}

    \begin{Verbatim}[commandchars=\\\{\}]
{\color{incolor}In [{\color{incolor}35}]:} \PY{k+kn}{from} \PY{n+nn}{hcluster} \PY{k}{import} \PY{n}{linkage}\PY{p}{,} \PY{n}{to\PYZus{}tree}
         \PY{k+kn}{from} \PY{n+nn}{ete3} \PY{k}{import} \PY{n}{Tree}
         
         \PY{k}{def} \PY{n+nf}{build\PYZus{}tree}\PY{p}{(}\PY{n}{table}\PY{p}{)}\PY{p}{:}
             \PY{c+c1}{\PYZsh{}hcluster part}
             \PY{n}{l} \PY{o}{=} \PY{n}{linkage}\PY{p}{(}\PY{n}{table}\PY{p}{,} \PY{l+s+s2}{\PYZdq{}}\PY{l+s+s2}{single}\PY{l+s+s2}{\PYZdq{}}\PY{p}{)} \PY{c+c1}{\PYZsh{} link the data}
             \PY{n}{T} \PY{o}{=} \PY{n}{to\PYZus{}tree}\PY{p}{(}\PY{n}{l}\PY{p}{)} \PY{c+c1}{\PYZsh{} create a tree}
         
             \PY{c+c1}{\PYZsh{}ete2 section}
             \PY{n}{root} \PY{o}{=} \PY{n}{Tree}\PY{p}{(}\PY{p}{)}
             \PY{n}{root}\PY{o}{.}\PY{n}{dist} \PY{o}{=} \PY{l+m+mi}{0}
             \PY{n}{root}\PY{o}{.}\PY{n}{name} \PY{o}{=} \PY{l+s+s2}{\PYZdq{}}\PY{l+s+s2}{root}\PY{l+s+s2}{\PYZdq{}}
             \PY{n}{item2node} \PY{o}{=} \PY{p}{\PYZob{}}\PY{n}{T}\PY{p}{:} \PY{n}{root}\PY{p}{\PYZcb{}}
             \PY{n}{names} \PY{o}{=} \PY{l+s+s2}{\PYZdq{}}\PY{l+s+s2}{ABCDEFGH}\PY{l+s+s2}{\PYZdq{}}
         
             \PY{n}{to\PYZus{}visit} \PY{o}{=} \PY{p}{[}\PY{n}{T}\PY{p}{]}
             \PY{k}{while} \PY{n}{to\PYZus{}visit}\PY{p}{:} \PY{c+c1}{\PYZsh{} While there are items to visit}
                 \PY{n}{node} \PY{o}{=} \PY{n}{to\PYZus{}visit}\PY{o}{.}\PY{n}{pop}\PY{p}{(}\PY{p}{)} \PY{c+c1}{\PYZsh{} take the first}
                 \PY{n}{cl\PYZus{}dist} \PY{o}{=} \PY{n}{node}\PY{o}{.}\PY{n}{dist} \PY{o}{/} \PY{l+m+mf}{2.0} \PY{c+c1}{\PYZsh{} take the cluster distance}
                 \PY{k}{for} \PY{n}{ch\PYZus{}node} \PY{o+ow}{in} \PY{p}{[}\PY{n}{node}\PY{o}{.}\PY{n}{left}\PY{p}{,} \PY{n}{node}\PY{o}{.}\PY{n}{right}\PY{p}{]}\PY{p}{:} \PY{c+c1}{\PYZsh{} fill in the binary sons}
                     \PY{k}{if} \PY{n}{ch\PYZus{}node}\PY{p}{:}
                         \PY{n}{ch} \PY{o}{=} \PY{n}{Tree}\PY{p}{(}\PY{p}{)} \PY{c+c1}{\PYZsh{} create a tree}
                         \PY{n}{ch}\PY{o}{.}\PY{n}{dist} \PY{o}{=} \PY{n}{cl\PYZus{}dist} \PY{c+c1}{\PYZsh{} set its distance}
                         \PY{n}{ch}\PY{o}{.}\PY{n}{name} \PY{o}{=} \PY{n}{names}\PY{p}{[}\PY{n}{ch\PYZus{}node}\PY{o}{.}\PY{n}{id}\PY{p}{]} \PY{k}{if} \PY{n}{ch\PYZus{}node}\PY{o}{.}\PY{n}{id} \PY{o}{\PYZlt{}} \PY{n+nb}{len}\PY{p}{(}\PY{n}{names}\PY{p}{)} \PY{k}{else} \PY{n+nb}{str}\PY{p}{(}\PY{n}{ch\PYZus{}node}\PY{o}{.}\PY{n}{id}\PY{p}{)} \PY{c+c1}{\PYZsh{} and name}
                         \PY{n}{item2node}\PY{p}{[}\PY{n}{node}\PY{p}{]}\PY{o}{.}\PY{n}{add\PYZus{}child}\PY{p}{(}\PY{n}{ch}\PY{p}{)} \PY{c+c1}{\PYZsh{} Add it as a child of parent node}
                         \PY{n}{item2node}\PY{p}{[}\PY{n}{ch\PYZus{}node}\PY{p}{]} \PY{o}{=} \PY{n}{ch} \PY{c+c1}{\PYZsh{} Set it as a node}
                         \PY{n}{to\PYZus{}visit}\PY{o}{.}\PY{n}{append}\PY{p}{(}\PY{n}{ch\PYZus{}node}\PY{p}{)} \PY{c+c1}{\PYZsh{} Visit it}
         
             \PY{c+c1}{\PYZsh{} This is your ETE tree structure}
             \PY{k}{return} \PY{n}{root}
\end{Verbatim}


    Finally we can build the tree and render it using render function of
ete3 tree node.

    \begin{Verbatim}[commandchars=\\\{\}]
{\color{incolor}In [{\color{incolor}58}]:} \PY{n}{tree} \PY{o}{=} \PY{n}{build\PYZus{}tree}\PY{p}{(}\PY{n}{tables}\PY{p}{[}\PY{l+s+s1}{\PYZsq{}}\PY{l+s+s1}{gzip}\PY{l+s+s1}{\PYZsq{}}\PY{p}{]}\PY{p}{[}\PY{l+m+mi}{1}\PY{p}{]}\PY{p}{)}
         \PY{n}{\PYZus{}} \PY{o}{=} \PY{n}{tree}\PY{o}{.}\PY{n}{render}\PY{p}{(}\PY{l+s+s1}{\PYZsq{}}\PY{l+s+s1}{tree1.png}\PY{l+s+s1}{\PYZsq{}}\PY{p}{,} \PY{n}{w}\PY{o}{=}\PY{l+m+mi}{1000}\PY{p}{,} \PY{n}{units}\PY{o}{=}\PY{l+s+s1}{\PYZsq{}}\PY{l+s+s1}{px}\PY{l+s+s1}{\PYZsq{}}\PY{p}{)}
\end{Verbatim}


    \subsection{Phylogenetic tree \#1 -
gzip}\label{phylogenetic-tree-1---gzip}

    \begin{Verbatim}[commandchars=\\\{\}]
{\color{incolor}In [{\color{incolor}59}]:} \PY{n}{tree} \PY{o}{=} \PY{n}{build\PYZus{}tree}\PY{p}{(}\PY{n}{tables}\PY{p}{[}\PY{l+s+s1}{\PYZsq{}}\PY{l+s+s1}{gen}\PY{l+s+s1}{\PYZsq{}}\PY{p}{]}\PY{p}{[}\PY{l+m+mi}{1}\PY{p}{]}\PY{p}{)}
         \PY{n}{\PYZus{}} \PY{o}{=} \PY{n}{tree}\PY{o}{.}\PY{n}{render}\PY{p}{(}\PY{l+s+s1}{\PYZsq{}}\PY{l+s+s1}{tree2.png}\PY{l+s+s1}{\PYZsq{}}\PY{p}{,} \PY{n}{w}\PY{o}{=}\PY{l+m+mi}{1000}\PY{p}{,} \PY{n}{units}\PY{o}{=}\PY{l+s+s1}{\PYZsq{}}\PY{l+s+s1}{px}\PY{l+s+s1}{\PYZsq{}}\PY{p}{)}
\end{Verbatim}


    \subsection{Phylogenetic tree \#2 -
GenCompress}\label{phylogenetic-tree-2---gencompress}


    % Add a bibliography block to the postdoc
    
    
    
    \end{document}
